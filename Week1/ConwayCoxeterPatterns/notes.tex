\documentclass{article}
\def\npart {2}
\def\nterm {LMS Summer School}
\def\nyear {2021}
\def\nlecturer {Sophie Morier-Genoud : Sorbonne Universite, Paris}
\def\ncourse {Arithmetic and combinatorics of Conway-Coxeter frieze patterns}

\input{header}

\begin{document}
  \maketitle

\section{Overview and }
\subsection{Continued fraction}
Let's take $x \in \R$ and we can say, $a < x < a + 1$ where $a \in \Z$. Now we write, $x = a + \frac{1}{x'}$, $x' > 1$.

\begin{align*}
  x = a + \frac{1}{x'} \\
  &= a + \frac{1}{a' + \frac{1}{x''}}\\
  &\vdots\\
  &= a_1 + \frac{1}{a_2 + \frac{1}{a_3 + \frac{1}{\ddots}}}
\end{align*}
where $a_1 \in \Z$ and $a_i \ge 1$ for $\forall i \ge 2$. \\

\begin{nlemma}
  The sequence of $a_i$'s are finite if and only if $x \in \Q$.
\end{nlemma}
\begin{proof}
  If it's finite, it's obvious that it's rational.\\
  Conversely, this is due to Euclid's algorithm. Take $\frac{p}{q} \in \Q$ and then write,
  \begin{align*}
    p &= a_1q + r_1 && 0 \le r_1 \le q\\
    q &= a_2r_1 + r_2 && 0 \le r_2 \le r_1\\
  \end{align*}
\end{proof}
or we can form them by subtracting instead of adding, these make negative expansion (or Hirzebruch-Sung),
$$ x = a_1 - \frac{1}{a_2 - \frac{1}{a_3 - \frac{1}{\ddots}}} $$

\begin{notation}
 $x = [a_1, a_2, \dots]$ means $x$ is a regular continued fraction and $x = [[a_1, a_2, \dots]]$ means a negative expansion.
\end{notation}

  Let's take $x \notin \Q$ and consider the golden ratio,\\
  $$ \varphi = [1, 1, 1, 1, 1, 1, 1, \dots] $$
  and now $e$, there is some regularity,
  $$ e = [2, 1, 2, 1, 1, 4, 1, 1, 6, 1, 1, 8, 1, \dots] $$
  and now $\pi$, there is no regularity,
  $$ \pi = [3, 7, 15, 1, 292, 1, 1, \dots] $$

Now consider the generalised continued fraction,
$$ \pi = \frac{4}{1 + \frac{1^2}{2 + \frac{3^2}{2 + \frac{5^2}{\ddots}}}} $$
Now consider the negative expansion,
$$ \![2, 2, 2, 2, 2, 2, \dots\!] $$
This is just one. It may be represented with an infinite sequence.\\

Here are some more examples,
\begin{align*}
  \frac{5}{2} &= [2, 2] = \![ 3, 2 \!]\\
  \frac{10}{3} &= [3, 3] = \![ 4, 2, 2 \!]\\
  \frac{14}{5} &= [2, 1, 3, 1] = \![ 3, 5 \!]\\
  \frac{43}{16} &= [2, 1, 2, 5] = \![ 3, 4, 2, 2, 2, 2 \!]
\end{align*}

\begin{remark}
  To make it unique one can always choose an even number for the length of the sequence. For a regular expansion, we can always assume that it's of even length.
\end{remark}

There is a very natural question, given $x = [a_1, \dots, a_{2m}] = \![c_1, c_2, \dots c_k \!]$, what is the relationship between $a_i$'s and $c_i$'s?

\subsection{Friezes}
Friezes are from Coxeter in the 1970's. We have the first and last row as 1's. When you have a diamond of numbers,

\begin{figure}[!ht]
\centering
\begin{tabular}{cccccccccccccccccccc}
 1&&1&&1&&1&&1&&1&&1&&1&&1&&1 \\
 &4&&1&&2&&2&&3&&1&&2&&4&&1&\\
 \dots&&3&&1&&3&&5&&2&&1&&7&&3&&\dots\\
 &5&&2&&1&&7&&3&&1&&3&&5&&2&\\
 \dots&&3&&1&&2&&4&&1&&2&&2&&3&&\dots\\
  &1&&1&&1&&1&&1&&1&&1&&1&&1& \\
\end{tabular}
\caption{A frieze pattern}
\end{figure}

and we also have the frieze patter rule, dubbed the unimodular rule, taking a part of the frieze we can create it by saying,
\begin{figure}[!ht]
  \centering
  \begin{tabular}{ccc}
    &b&\\
    a&&d\\
    &c&\\
  \end{tabular}
\end{figure}

and we restrict the entries such that $ad - bc = 1$.
%     b
%  a    d
%     c
% ad - bc = 1

\begin{nthm}[Coxeter]
  Friezes are periodic!
\end{nthm}

We shall consider the width of the frieze, don't count the ones. The example has width of 4 and we can look at the width + 4, we get some perdiodic notion. This means we just need the first `row' to fill the frieze.\\

Furthermore, we now have some sort of glide symmetry. We can form triangles and they repeat themselves. Let's finish the statement

\begin{nthm}[Coxeter]
  Friezes are periodic!
  \begin{itemize}
    \item Width $w \to (w+3)$ is a period.
    \item Invariant under a glide symmetry.
  \end{itemize}
\end{nthm}

\begin{ex}
  Exerises 1,2,3 and 4.
\end{ex}

\textbf{Question 2:} How to get frieze containing only positive integers?\\

\subsection{Triangulation of n-gons}

\begin{ndefi}[Diagonal]
  A line that joins two non-consecutive vertices.
\end{ndefi}

\begin{ndefi}[Triangulation]
  The maximum number of non-intersecting diagonals.
\end{ndefi}
The triangulation is not unique. The number of triangulations is related to the Catalan numbers. For every triangulation, you may two or more triangles on the exterior. We will consider exactly two exterior triangles for the minute,\\
Fix a triangulation with two exterior triangles, if you join two external vertices of these external triangles, then you will intersect every other triangle.\\

If you consider the triangles layed next to eachother like this,
\begin{figure}[!ht]
  \centering
  \includegraphics{./figures/L1.2}
  \caption{}
\end{figure}
We can count how many triangles are incidence to the vertex. We shall call this $c_i$ where $i$ is the number of the vertex. If we do this for the 7-gon,
\begin{figure}[!ht]
  \centering
  \includegraphics{./figures/L1.3}
  \caption{}
\end{figure}
The sequence $(a_1, a_2, \dots, a_{2n})$ determines the triangulation and the so does the sequence, $(c_1, c_2, \dots)$.

\begin{nthm}
  With this data, we can say that,
  \begin{itemize}
    \item $[a_1, a_2, \dots, a_{2m}] = \![ c_1, c_2, \dots, c_k \!]$
    \item $(c_1, c_2, \dots, c_n)$ determines a frieze of width $n-3$ containing only positive integers.
  \end{itemize}
\end{nthm}

\begin{remark}
  On 1) this is called this Hirzebruch Formula,
  $$ [a_1, a_2, a_3, \dots, a_{2m}\,] = \![ a_1+1, 2, 2, \dots, 2, a_3+2, 2, \dots \,\!] $$
\end{remark}

\begin{remark}
  On 2) It works with any kind of triangulations and all friezes arise this way. (This is Conway-Coxeter Theorem) and a Conway-Coxeter Frieze is a frieze with positive integers.
\end{remark}

\section{Proving Conway-Coxeter Theorem}
Our main theorem is,
\begin{nthm}[Conway-Coxeter]
  There is a bijection,
  $$ \{\text{freizes of $\Z_{>0}$}\} \longleftrightarrow \{\text{Triangulations of polygons with $w + 3$ polygons} \} $$
\end{nthm}
This is given by,
$$ (c_1,c_2,\dots,c_{w+3}) \longleftrightarrow c_i = \text{\# of triangles incident to $i$} $$

\begin{ndefi}[Quiddity Sequence]
  We call a quiddity sequence of a polygon, just the $c_i$'s
\end{ndefi}

This theorem is from `Triangulated Polygons and Frieze patters', it is a two part paper. The first asks questions about friezes, 33 of them. then they give the questions later. The result is lost in the questions. Coxeter gave all credit to Conway. Conway was a victim to the pandemic, RIP. \footnote{I loved Conway's work, what a legend. I missed meeting him once, by five minutes.}
Conway called the friezes and the result a miricle and doesn't quite know how he came up with them.

\begin{eg}
  We can look at the following for $w=1$,
  \begin{figure}[!ht]
  \centering
  \begin{tabular}{cccccccccc}
   1&&1&&1&&1&&1& \\
   &$x$&&$\frac{2}{x}$&&$x$&&$\frac{2}{x}$&&\dots\\
   1&&1&&1&&1&&1& \\
  \end{tabular}
  \end{figure}
  It has two solutions of $x = 1$ and $x = \frac{1}{2}$, hence we have two friezes,
  $$ \begin{matrix}
    1 & 2 & 1 & 2
  \end{matrix} $$
  and,
  $$ \begin{matrix}
    2 & 1 & 2 & 1
  \end{matrix} $$
  These are two different triangulations. One with vertex 2 and 4 connected and the other 1 and 3 connected.
\end{eg}

and for $w = 2$, we have
\begin{eg}
  We have a different constraint because we want positive integers, this is exercise 2 and it has 5 solutions.
  \begin{figure}[!ht]
  \centering
  \begin{tabular}{cccccccccc}
   1&&1&&1&&1&&1& \\
   &$x_1$&&$\cdot$&&$\cdot$&&$\cdot$&&\\
   &&$x_2$&&$\cdot$&&$\cdot$&&\\
   &1&&1&&1&&1&& \\
  \end{tabular}
  \end{figure}
  If we look at pentagons we have the same thing by cyclic permutations up to quiddity.
  $$ (x_1, x_2) = (1, 1) ,(1, 2), (2, 1), (2, 3), (3, 2) $$
\end{eg}

If we consider the possible case where $w = 0$, we get a degenerate frieze.

\begin{eg}
  This produces a frieze of `no filling'.
  \begin{figure}[!ht]
  \centering
  \begin{tabular}{cccccccccc}
   1&&1&&1&&1&&1& \\
   &1&&1&&1&&1&& \\
  \end{tabular}
  \end{figure}
\end{eg}

\begin{nthm}[Thm 1]
  \begin{itemize}
    \item Friezes of width $w$ are $(w + 3)$-periodic
    \item Friezes are invariant uder a glide reflection
    \item Let $c_1, c_2, \dots c_{w+3}$ be the entries in the first row, then all the entries are polynomials in $c_i$'s.
    \item Let $x_1, x_2,\dots, x_w$ be the entries in a zig-zag from top to bottom in the frieze, then all the entries are Laurent polynomials in $x_i$'s with coefficient in $\Z_{>0}$
  \end{itemize}
\end{nthm}

Comments:
\begin{itemize}
  \item Item 2 implies 1, as a glide reflection is just a periodic flip and translation.
  \item Item 3, look at Exercise 3.
  \begin{figure}[!ht]
    \centering
    \includegraphics{./figures/L2.4}
  \end{figure}
  \item Item 4,
  \begin{ndefi}[Laurent Polynomial]
    $\frac{P(x_1, x_2, x_3, \dots x_w)}{x_1^{k_1}x_2^{k_2}\dots x_w^{k_w}}$
  \end{ndefi}
  and if consider a zig zag, all we are going os just saying we are considering things to the left or right and below your current position.
\end{itemize}

\begin{proof}[Proof of Thm 1]
  Start with a Conway-Coxeter frieze and call the values of the first row $\dots, c_0, c_1, \dots, (c_i)_{i\in\Z}$. We fix three consecutive diagonals.
  \begin{figure}[!ht]
    \centering
    \includegraphics{./figures/L2.5}
  \end{figure}
  \newpage
  \begin{lemma}
    $\frac{f_i + h_i}{g_i} = K$
  \end{lemma}
  \begin{proof}
    Apply the frieze rule round these values and get,
    \begin{align*}
      1 &= g_i f_{i+1} - f_ig_{i+1}\\
      1  &= h_ig_{i+1} - g_ih_{i+1}\\
        g_if_{i+1} + g_ih_{i+1} &= h_ig_{i+1} + f_{i}g_{i+1}\\
        \frac{f_{i+1}h_{i+1}}{g_{i+1}} &= \frac{f_{i}h_{i}}{g_{i}}
    \end{align*}
    and so we can see,
    \begin{align*}
      \frac{f_0+h_0}{g_0} &= \frac{f_{w+1} + h_{w+1}}{g_{w+1}}\\
      c_0 &= f_{w+1}
    \end{align*}
    and so we have that we have a type of glide symmetry.
  \end{proof}
  Now using identical argument, we can start to determine more of the frieze. Hence, we have proved the glide symmetry. Now, we have also proved (1), which is implied by (2).\\

  We are going to take diagonals in the other direction and use similar arguments, we will find that,
  $$ \frac{h_0+h_2}{h_1} = \frac{g_0 + g_2}{g_1} = \frac{f_0 + f_2}{f_1} = c_2 $$
  and hence, we obtain the recurrance relation of,
  $$ g_i = c_ig_{i-1} - g_{i-2} $$
  This is the key lemma for the big proof. If $g_{i-1}$ and $g_{i-2}$ are polynomials in $c$, then so is $g_i$. Now, we have proved (3). \\

  Finally, for (4), exercise 4 gives a proof and an explicit formula in the particular case of a diagonal. In the case of an arbitrary zig-zag, we don't know any elementary proof, the one I know is based on Cluster Algebra and Cluster Algebra. You can focus on elements of the frieze as cluster elements.
\end{proof}

\end{document}
